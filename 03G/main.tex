\documentclass{article}
\usepackage{../fasy-hw}
\usepackage{ wasysym }

%% UPDATE these variables:
\renewcommand{\hwnum}{3}
\author{Gavin Austin, George Engel, Troy Oster}
\collab{None}
\date{due: 18 October 2019}

\begin{document}

\nextprob
If 23 people are in a room, then the probability that at least two of them have
the same birthday is at least one half.  This is known as the birthday paradox,
since the number 23 is probably much lower than you would expect.  How many
people do we need in order to have 50\% probability that there are three people
with the same birthday?

We need 87 people in order to have 50\% probability that there are three people with the same birthday.
\newline \newline Let $P(A) = $ the probability that at least three people in the room have the same birthday.
\newline Then, $P(A') = $ the probability that no three people in the room have the same birthday.
\newline Therefore, $P(A) = 1 - P(A')$.
\newline \newline $P(A') = (1/365)^{87} \,\cdot\, (365 \cdot 364 \cdot 363 \cdot \, ... \, \cdot 279)$
\newline $P(A) = 1 - 0.48893488937 = 0.5110651106$
\newline $P(A) = 51.10651106\%$


\nextprob
Suppose we have a graph $G=(V,E)$ and three colors, and randomly assign a color
each node (where each color is equally likely).
\begin{enumerate}
    \item What is the probability that every edge has two different colors on
        assigned to its two nodes?
    \item What is the expected number of edges that have different colors
        assigned to its two nodes?
\end{enumerate}

\nextprob
Suppose we have a graph $G=(V,E)$ and three colors, and randomly assign a color
each node (where each color is equally likely).
\begin{enumerate}
    \item What is the probability that every edge has two different colors on
        assigned to its two nodes?
    \item What is the expected number of edges that have different colors
        assigned to its two nodes?
\end{enumerate}

\nextprob
CLRS, Question 15-6.

\boldsymbol{Explanation of the Problem:}

Question as described in book: Describe an algorithm to make up a guest list that maximizes the sum of the conviviality ratings of the guests. Analyze the running time of your algorithm.

Problem broken down: 

We want to get a subset of employees $E$ that maximizes the sum of conviviality between employees not including the employee's immediate superior.

So if we think of each employee as a node $e$, and denote the employee's value of conviviality as $e[c]$, we would want to find the largest value for $$\sum_{e \in E}^{}(e[c])$$ such that if $e \in E$, then e's parent \notin E.

\boldsymbol{Explanation of Algorithm:}

Given we want to find the optimal solution to a summation, it would make sense to try and beak down the problem of finding the optimal solution into smaller sub problems, using dynamic programming.

1) If e is the president (the root of our sub tree), and is invited, let the maximum possible conviviality sum be denoted as M(e).

2) Else, e is the root of our sub tree, but is not invited, let the maximum possible conviviality sum be denoted as M'(e).

3) Say M(e) = The sum of each generation of employees, disregarding every other child generation, including/starting at the root node.

4) Say M'(e) = The sum of each generation of employees, disregarding every other child generation, not including/starting at the root node.

5) For the solution, we begin with the company president, and recursively traverse the company hierarchy  CONTINuE

\boldsymbol{Psuedocode}

Computing M(x) and M'(x) for every employee e using a single recursive traversal of the company tree in O(n) time.

\begin{algorithm}
    \caption{Calculating Maximum Conviviality}
    \begin{algorithmic}
        \Procedure{Party\_Planner}{$e$} 
            \State M(e) = e[c]
            \State M'(e) = 0
            \State child = e.leftmost\_child
            \While{ child $\neq$ NIL} 
                \State Party\_Planner(child)
                \State M(e) = M(e) + M'(e)
                \State M\_child = M(child)
                \State M\_not\_child = M'(child)
                \If{M\_child $>$ M\_not\_child} \Comment{We want to add the larger value}
                    \State M'(e) = M'(e) + M\_child
                \Else
                    \State M'(e) = M'(e) + M\_not\_child
                \EndIf
                
                \State child = child.get\_sibling\_to\_right
            \EndWhile
            
            
            \If{}
                \State M'(e) = M'(e) + M\_child
            \Else
                \State M'(e) = M'(e) + M\_not\_child
            \EndIf
        \State return conviviality
        \EndProcedure
    \end{algorithmic}
\end{algorithm}


\begin{algorithm}
    \caption{Actually Inviting Employees}
    \begin{algorithmic}
        \Procedure{Invite}{$e$} 
            \State invited = [] \Comment{List of employees invited to party.}
            \If{M(e) $>$ M'(e)} 
                    \State invited.append(e)
                    \For{each grandchild of e}
                        \State invited.append(Invite(grandchild))
                    \EndFor
                \Else
                    \For{each child of e}
                        \State invited.append(Invite(child))
                    \EndFor
                \EndIf
                \State return invited
        \EndProcedure
    \end{algorithmic}
\end{algorithm}

\boldsymbol{Decrementing Function:}

TODO: This.

\boldsymbol{Run\ Time\ Justification:}

Our algorithm's run time is $O(n)$. 

TODO: CHANGE THIS!!!

Assuming there are n vertices, given we solve n sub problems each of which are constant time. However, it is worth noting that the it could take linear time to find the next node to solve using the tree.

\boldsymbol{Loop Invariants:}

TODO: This.

\nextprob
For the Greedy make change algorithm described in class on 10/02, describe the
problem and solution in your own words, including the use of pseudocode (with
more details than what was written in class).  Note: you do not need to give a
loop invariant and the proof of termination/runtime complexity.

The Problem:

Given a set of coin values representing a currency, find the minimum set of available coins that add up to the 'change' value.

The solution:

1) First, we initialize the resulting list of coins as empty.

2) sort the list of coin values for the current currency.

3) Then we take the largest coin value for the current currency list that is smaller then our 'remaining\_change' value.

4) Add that largest coin to the result list, and set our 'remaining\_change' value equal to our 'remaining\_change' minus that largest coin's value.

5) As a base case, if 'remaining\_change' becomes 0, then we print out/return the resulting list of coins. Otherwise we repeat sets 3 and 4 with the new value of 'remaining\_change'. 

NOTE: We assume there is a one cent equivalent value such that the base case can always occur.

\begin{algorithm}
    \caption{Greedy Make Change Pseudo Code}
    \begin{algorithmic}
        \Procedure{Greedy\_Make\_Change}{$currency\_list$, $remaining\_change$} 
            \State $result\_list$ = []
            \If{$remaining\_change$ $==$ $0$}
                \State return $result\_list$
            \EndIf
            
        \State sort(currency\_list, LARGE\_TO\_SMALL) \Comment{Sorted order Large to Small by value.}
        
        \For{$i$ = 0; i $<$ $currency\_list.length$; $i++$}
            \If{$currency\_list[i].value()$ $<$ $remaining\_change$}
                \State $result\_list.append(currency\_list[i])$ \Comment{Add the first value smaller than} $remaining\_change$}
                \State $remaining\_change$ $-=$ currency\_list[i].value()
                \State Break
            \EndIf
        \EndFor
        
        \State $result\_list$.concatenate($Greedy\_Make\_Change(currency\_list$, $remaining\_changes)$)
        
        \State \textbf{return} result\_list
        \EndProcedure
    \end{algorithmic}
\end{algorithm}

\nextprob
For the Greedy make change algorithm described in class on 10/02, describe the
problem and solution in your own words, including the use of pseudocode (with
more details than what was written in class).  Note: you do not need to give a
loop invariant and the proof of termination/runtime complexity.

\nextprob
Suppose we have $n$ items hat we want to put in a knapsack of capacity $W$.  The i-th item has
weight $w_i$ and value $v_i$.  The knapsack can hold a total weight of $W$ and
we want to maximize the value of the items in the knapsack.
The \emph{0-1 knapsack problem} will assign each item one of two states: in the
knapsack, or not in the knapsack.  The \emph{fractional knapsack problem} allows
you to take a percentage of each item.
\begin{enumerate}
    \item Give an $O(n\log n)$ greedy algorithm for the fractional knapsack problem.
    \item Give an $O(nW)$ time algorithm that uses dynamic programming to solve
        the 0-1 knapsack problem.
\end{enumerate}

\end{document}
